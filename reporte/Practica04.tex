% Tipo
\documentclass{article}

% Márgen
\usepackage[margin = 1.5cm]{geometry}

% Símbolos
\usepackage{amsmath}

%Imágenes
\usepackage{graphicx}

%Idioma
\usepackage[spanish]{babel}
\usepackage[utf8]{inputenc}

\begin{document}
    \title{
        Organización y Arquitectura de Computadoras \\
        2019-2 \\
        Práctica 4: Unidad Aritmético Lógica
    }
    \author{
        Sandra del Mar Soto Corderi \\
        Edgar Quiroz Castañeda
    }
    \date{
        3 de marzo del 2019
    }

    \maketitle

    \section{Preguntas}

    \begin{enumerate}
    %1
        \item 
            ¿Qué operaciones aritmeticas y lógicas son básicas para un procesador?
Justifica tu respuesta.\\

De operaciones aritmeticas, las básicas son:
\begin{itemize}
\item
Suma: la suma es las operación más importante de todas, ya que se emplea para el cálculo de la dirección de la siguiente instrucción, se utiliza para el cálculo de las direcciones a los operandos y otras operaciones la emplean
\item
Resta: Ya que es una de las operaciones aritméticas más importantes, y porque manejamos números inversos
\item
Multiplicación: Ya que permite crear después operaciones más complejas
\item
División: Es el inverso de la multiplicación
\end{itemize}

De operaciones lógicas, las básicas son:
\begin{itemize}
\item
AND
\item
OR
\item
NOT
\end{itemize}
Estas 3 son necesarias, ya que a partir de ella podemos crear cualquier compuerta lógica y podemos expresar cualquier operador.\\				

%2		
		\item
		El diseño utilizado para realizar la adición resulta ser ineficiente, ¿Por qué? ¿Qué tipo de sumador resulta ser más eficiente?
		
		Es ineficiente porque tenemos que detectar el caso de uso, un sumador donde el número de sumas y resta esta asociado a las transiciones de ceros y unos, se puede
demostrar que resulta en un método más eficiente, pues el número promedio de estas
transiciones es menor que el número promedio de unos presentes en una palabra.\\
		
	
%3	
		\item
		Bajo este diseño, en la ALU se calculan todas las operaciones de forma
simultanea pero sólo se entrega un resultado, ¿Se realiza trabajo inútil?
¿Toma tiempo adicional? ¿Cuál es el costo?\\

Es un sistema bastante barato para las operaciones que realiza, toma mayor tiempo que un sistema determinado a una de las operaciones en particular y realiza trabajo innecesario como tener el acarreo cuando utilizamos funciones lógicas como AND u OR.\\


%4
		\item
		¿Cuántas operaciones más podemos agregar al diseño de esta ALU? ¿Qué
tendríamos que modificar para realizar más operaciones?\\

		Podríamos agregar multiplicación y división si modificamos partes del circuito y le agregamos circuitos más complejos, como por ejemplo para la multiplicación, necesitariamos agregar un circuito como en la siguiente imagen:\\
		
		\includegraphics[scale=0.5]{Multiplicador.jpg}
		          
            
    \end{enumerate}
  
    
    
\end{document}